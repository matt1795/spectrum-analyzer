\section{Goals}

The main purpose/goals for this project are:

\begin{itemize}
	\item{Hands on experience with RF circuits}
	\item{Design PCB Layout for RF Circuits}
	\item{Get a sweet SA}
	\item{Catalyst for even more RF work}
\end{itemize}


\subsection{Long-Term}

So once all of this is done and I've got a halfway decent spectrum analyzer
designed, I want to create a digital interface for use on a linux based machine.
This would also be a large undertaking since the analog to digital interface is
a module unto itself, and I will likely get a reason to finally make a linux
kernel driver, which I've been wanting to do for a while. And then from there
the benefits for experimentation would be awesome, a programming interface for
testing and access to more data for all sorts of computations.

\section{Design}

\begin{circuitikz} \draw
	(0,0) to[short,o-] ++(0.3,0)
	to[lowpass,>] ++(2,0)
	node[mixer] (m) {}
	(m.1) to[short] ++(-1,0)
	(m.2) to[vco] ++(0,-1);
\end{circuitikz}

The overall system design comes mostly from \cite{qsl} and I will be extensively
experimenting with each module's implementation.

\subsection{Power Supply}

\cite{qsl} uses +/- 15V supply rails, says to only used linear regulators

I want to test with switching mode power supplies to see the effect.

\subsection{Step Attenuator}

\cite{qsl} Hittite HMC307 (31dB eval board)

\subsection{Superheterodyne Receiver}

\subsubsection{Input Low-Pass Filter}

1GHz Low Pass filter

\cite{qsl} 3dB point at 1GHz, 0.65 dB insertion loss, equivalent from
mini-circuits: SLP-1000

\cite{lea} had a 1.75GHz SA and used a microstrip filter for this. I do not have
any size requirements, so I might attempt to create a 1GHz microstrip filter.
The wavelength is

\begin{equation}
	\lambda = \frac{c}{f}
	= \frac{3*10^8 m/s}{1GHz}
	= 0.3m
\end{equation}

which might be a bit large, but if I only have to keep components to 1/2 or 1/4
wavelength then that just means 15cm and 7.5cm respectively.

Using the SLP-1000 as a baseline for attenuation performance, the 20dB point is
as 1.35 times the cuttoff frequency (it says 990MHz is the 3dB cutoff but let's
round up to 1GHz), and the 40dB point at 1.75. This roughly corresponds to a
rolloff of 150 and 175 dB/decade respectively. Hooo fuck bud, that's some high
order filters right there.

More details of this filter's implementation will be discussed in a later report
where we get our hands dirty.

\subsubsection{First Mixer}

\cite{qsl} "ideal" will have 40dB isolation between LO and RF ports, make sure
that it has good low frequency response.

\cite{lea} states that the low frequency response of the filter is based on the
noise figure while the higher frequency response is determined by its distortion
-- intermods.

\cite{lea} argues that the mixer has little protection from DC or overpowered
RF, so it should be able to be repaired easily. Commercially available ones are
doubly balanced with schottky quads and ferrite balancing transformers. Instead
the simple approach is going to use a double schottky diode -- BAT14-099.

\subsubsection{First Local Oscillator}

\subsubsection{Cavity BPF Filter}

\cite{lea} had a BPF Filter center frequency of 2.1GHz and a bandwidth of 25MHz.

\begin{equation}
	Q = frac{f_o}{BW}
	= frac{2.1GHz}{25MHz}
	= 84
\end{equation}

Both \cite{qsl} and \cite{lea} used cavity filters, and the high frequency of
operation likely dispells the used of linear compontents. While the microstrip
filters are really cool, after some reading I found that the downside to them is
that the pcb material contributes to dielectric losses, and since this filter
needs to be tight as fuck, a cavity filter will do.

\subsubsection{Second Mixer}

\subsubsection{Second Local Oscillator}

\subsubsection{Post-Mixer Amp}

\subsection{Sweep and Sync}

\subsection{Output Stage}

\subsubsection{Resolution Filters}

While having a fixed resolution, or discrete 

\subsubsection{RF Switches}

\subsubsection{IF Amp}

\subsubsection{Log Detector Video Amp}

\section{Next Steps}

I plan on experimenting and designing the modules one after the other as I go,
making sure to do integration tests to make sure that I get expected system
performance once multiple modules are added togther.

Each lab will be focused on implementing a single module, 
